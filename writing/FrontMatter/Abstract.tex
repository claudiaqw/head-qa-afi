\chapter*{Resumen}\label{chapter:resumen}

Los sistemas Q/A son un paso crucial en el entendimiento del lenguaje natural. Este problema ha ganado gran atención desde el inicio de la linguística computacional, como consecuencia, varios métodos han sido propuestos para resolver esta tarea en el idioma inglés, los más recientes, aplicando en los el enfoque de aprendizaje profundo. Sin embargo, los avances en el idioma español siguen siendo escasos. Head-QA es un conjunto de datos de preguntas de selección múltiple orientado al dominio biomédico ... En este trabajo se proponen modelos basados en las redes profundas para la resolución del problema de selección de respuestas sobre la base del corpus Head-QA. Estos modelos no utilizan características léxicas y sintácticas como árboles de dependencia, partes de la oración y reconocedores de entidades nombradas, solo tienen en cuenta la posición de las entidades en una oración. Se presentan, comparan y discuten los modelos matemáticos asociados y las arquitecturas respectivas. Finalmente, se presenta un conjunto de experimentos orientados a la validación de los modelos de aprendizaje profundo que demuestran la efectividad de la solución.

\newpage
\chapter*{Abstract}\label{chapter:abstract}