%===================================================================================
% Chapter: Introduction
%===================================================================================
\chapter*{Introducción}\label{chapter:introduction}
\addcontentsline{toc}{chapter}{Introducción}
%===================================================================================

Desde tiempos inmemoriales, el ser humano procesa la información con el fin de descubrir conocimiento. Con el surgimiento de la Web y el avance de las tecnologías digitales, la generación de información ha experimentado un crecimiento desmesurado. El gran volumen existente actualmente hace imposible procesar toda la información manualmente al mismo ritmo que se genera. Como resultado, existe una gran acumulación de información que está siendo desaprovechada en términos del descubrimiento de  conocimiento. Por esta razón, una alternativa es recurrir a la extracción automática de información con el fin de reconocer la información relevante y organizarla de manera que pueda ser procesada por una computadora. Actualmente uno de los retos consiste en que la mayor parte del contenido presente en la Web es de naturaleza no estructurada pues se genera principalmente en forma de texto.

El lenguaje humano es increíblemente complejo y diverso. Nos expresamos de disímiles maneras, verbalmente y por escrito. No solo existen cientos de idiomas y dialectos, sino que en cada idioma existe un conjunto único de reglas gramaticales y de sintaxis, términos y palabras coloquiales. Entender, interpretar y manipular el lenguaje humano no es tarea fácil para una computadora.

La meta general es tomar texto del lenguaje y aplicar la lingüística y algoritmos para transformar o enriquecer el texto de tal forma que provea un mayor valor. Simulando la lógica humana, la clave consiste en localizar los aspectos importantes de la información y estructurar la información de manera que pueda ser manipulada por una computadora. A este proceso de transformar la información no estructurada en estructurada se le conoce como extracción de información y es un paso crucial en el procesamiento del lenguaje humano.

TODO

Este es un hecho que la comunidad científica no tardó en percibir, por lo cual muchos esfuerzos han sido dedicados a resolver esta tarea. Aunque todavía no se considera un problema resuelto, lo cierto es que se han presentado numerosas propuestas que se han superado en el tiempo, fundamentalmente en el idioma inglés. 


El \textbf{objetivo general} consiste en desarrollar un modelo pregunta/respuesta para el idioma español fundamentado científica y tecnológicamente.

Para lograr el cumplimiento del objetivo general, se plantea a continuación un conjunto de objetivos específicos:

\begin{enumerate}
    \item Profundizar desde los puntos de vista teórico-conceptual y práctico en el área de los sistemas pregunta/respuesta. 
    \item Concebir y diseñar un algoritmo que utilice el enfoque supervisado. Diseñar e implementar modelos de aprendizaje profundo con diferentes arquitecturas de red para dar respuesta a las preguntas.
    \item Evaluar cualitativa y experimentalmente los modelos propuestos y comparar los resultados de la soluciones concebidas.
\end{enumerate}

